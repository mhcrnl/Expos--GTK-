\documentclass{beamer}
\usepackage[french]{babel}
\usepackage[utf8]{inputenc}
\usepackage[T1]{fontenc}
\usetheme{Warsaw}
\title[GTK+]{GTK+\\Une boîte à outils pour interfaces graphiques}
\author{Flavien Godefroy et Kevin Bradshaw}
\date{18 Décembre 2012}
\begin{document}

\begin{frame}
\titlepage
\end{frame}

\AtBeginSubsection[]
{
  \begin{frame}<beamer>
    \frametitle{Layout}
    \tableofcontents[currentsection,currentsubsection]
  \end{frame}
}

\section{Introduction}

\subsection{Qu'est-ce?}
\begin{frame}{Quoi?}
GTK+ c'est GTK, mais en un peu plus mieux...
\end{frame}

\subsection{Qui s'en sert?}
\begin{frame}{Qui?}
GTK+ c'est pour les gens qui veulent faire des trucs où il y a des trucs
\end{frame}


\section{Histoire}
\subsection{GIMP}
\begin{frame}{GIMP}
GIMP est à l'origine GTK
\end{frame}

\subsection{Et plus!}
\begin{frame}{Le reste}
Mais bon...
\end{frame}


\section{Comparaison à ses competiteurs}
\subsection{Qt}
\begin{frame}{Qt}
Fucking Qt quoi...
\end{frame}

\section{Un petit projet avec GTK+}
\subsection{Calculatrice}
\begin{frame}{Finger in the nose!}
La classe ultime
\end{frame}


\section{Pourquoi GTK+?}
\subsection{Bonne question}
\begin{frame}{La réponse?}
Je n'en sais guère...
\end{frame}

\end{document}